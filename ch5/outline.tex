% !TEX program=lualatex
\RequirePackage{luatex85}
\documentclass{report}
\usepackage{geometry}
\usepackage[style=nature, citestyle=authoryear]{biblatex}
\usepackage{amsmath}
\usepackage{url}
\usepackage{graphicx}
\usepackage{tikz}
\usepackage{wrapfig}
\usepackage{booktabs}
\usepackage{tabularx}
\usepackage{multirow}
\usepackage{minted}

%drafting packages
\usepackage[doublespacing]{setspace}
\usepackage{lineno}

\newcommand{\titlecaption}[2]{\caption[#1]{\textbf{#1} \textbar\, #2}}
\bibliography{ch5.bib}
\graphicspath{{./figures/}}

\begin{document}
\linenumbers

\chapter{Evaluating the impact of quality score calibration on variant calling}

\section{Introduction}
\begin{outline}
\item Brief overview of DNA sequencing; erroneous bases
\item Variant calling
	\begin{outline}
	\item why is it used?
	\end{outline}
\item Quality scores, Base Quality Score Recalibration, and quality score binning
\end{outline}

\section{Methods}
\begin{outline}
\item Call variants with recalibrations of different quality
\item Call variants with binned quality scores, with differing number of bins
% if I have time, simulate some quality score differences and then call variants with those
\item Evaluate variants
\item Use KBBQ on the \textit{E. melliodora} data to call variants
\item Use the tree structure to evaluate the variant calls
\end{outline}

\section{Results}
\begin{outline}
\item Call variants with recalibrations of different quality
\item Call variants with binned quality scores, with differing number of bins
% if I have time, simulate some quality score differences and then call variants with those
\item Evaluate variants
\item Use KBBQ on the \textit{E. melliodora} data to call variants
\item Use the tree structure to evaluate the variant calls
\end{outline}

\section{Discussion}
\begin{outline}
\item As the quality of the calibration increases, the quality of the variant calls also slightly increases.
\end{outline}

\section{Conclusion}
\begin{outline}
\item Quality scores have a limited effect on the variants called
\item Excessive binning removes quality score information and makes it difficult to call variants
\item Use

\end{outline}

\printbibliography
\end{document}

