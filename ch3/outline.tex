% !TEX program=lualatex
\RequirePackage{luatex85}
\documentclass{article}
\usepackage{amsmath}
\usepackage[letterpaper,margin=1in]{geometry}
\usepackage{url}
\usepackage{graphicx}
\usepackage[utf8]{inputenc}
\usepackage[T1]{fontenc}
\usepackage[style=nature, citestyle=authoryear]{biblatex}
\usepackage{tikz}
\usepackage{wrapfig}
\usepackage{lineno}
\usepackage{outline}
\usepackage{tabularx}
\bibliography{citations.bib}
\graphicspath{./figures/}

\begin{document}
\linenumbers

\section{Introduction}
\begin{outline}
	\item Genotyping is a common task in biology.
	\begin{outline}
		\item Genotyping accurately and sensitively is difficult due to errors in sequencing reads.
		\item This problem is exacerbated when genotyping non-model organisms because it may lack a polished annotated genome and other prior information that is sometimes used in genotyping pipelines, such as base quality score recalibration.
	\end{outline}
	\item Reference-free variant callers
	\begin{outline}
	\end{outline}
	\item I previously called genotypes in the non-model organism \textit{Eucalyptus melliodora} using both out-of-the-box and custom methods to genotype leaves sampled across the tree. I leveraged the sample structure to estimate false positive and false negative rates for each step.

\end{outline}

\section{Methods}
\begin{outline}
	\item To understand how each filtering step affected the false positive and false negative rates, I estimated these rates at each step in the filtering process and evaluated how changes to filter paramaters affected the resulting phylogenies constructed using the resultant genotypes.
	\item To evaluate the performance of NON-REF CALLER and investigate why its performance was so low on this data, I applied the same procedure to the results of those variant calls.
\end{outline}

\section{Results}
\begin{outline}
	\item Trees from each step for emel data
	\item A table of filtering steps/false pos/false neg at each step for non-ref caller
	\item A tree for each filter step for variants from non-ref caller
\end{outline}

\section{Discussion}
\begin{outline}
	\item High false positive rate and false negative rate severely impacted the ability of the non-ref caller to accurately identify mutations
	\item X was the step that resulted in the best false pos/ false neg improvement for the reference-free caller. Compare this for best step in traditional pipeline.
\end{outline}

%\printbibliography

\end{document}